\documentclass[pdftex, smaller]{beamer}
\usepackage{graphicx}
\usepackage{listings}

\setlength{\rightskip}{0pt} % I want justifed text
\sloppy

\usepackage[utf8]{inputenc}
\usepackage[T1]{fontenc}

\newcommand{\F}[1]{{\protect\mathcal{F}_{#1}}}
\newcommand{\G}[1]{{\protect\mathcal{G}_{#1}}}
\newcommand{\U}[1]{{\protect\mathcal{U}_{#1}}}
\newcommand{\V}[1]{{\protect\mathcal{V}_{#1}}}
\DeclareMathOperator{\mesh}{mesh}
\DeclareMathOperator{\ulim}{ulim}
\DeclareMathOperator{\asdim}{asdim}
\DeclareMathOperator{\udim}{udim}
\DeclareMathOperator{\mult}{mult}

\mode<presentation> {
  \usetheme{Berkeley}
  \setbeamercovered{highly dynamic}
}

\usepackage[english]{babel}
\usepackage[utf8]{inputenc}
\usepackage{times}
\usepackage[T1]{fontenc}

\lstset{language=C++,
                basicstyle=\ttfamily\footnotesize,
                keywordstyle=\color{blue}\ttfamily,
                stringstyle=\color{red}\ttfamily,
                commentstyle=\color{green}\ttfamily,
                morecomment=[l][\color{magenta}]{\#}
}

\author{ZPK 2016}

\title[Obiekty i klasy]
{Obiekty i klasy}

\date{8 marca 2017}

\begin{document}

\begin{frame}
  \titlepage

  \begin{center}
    oraz paradygmat programowania obiektowego.
  \end{center}
\end{frame}

\begin{frame}
\frametitle{Zanim zaczniemy}

Kilka uwag o stylu kodowania.

\vspace{2mm}
\url{http://courses.cms.caltech.edu/cs11/material/cpp/donnie/cppstyle.html}

\vspace{2mm}
\begin{itemize}
\item Prawidłowo rozmieszczaj spacje
\item Konsekwentnie rozmieszczaj nawiasy klamrowe
\item Używaj nawiasów klamrowych jeżeli instrukcja zajmuje więcej niż jedną linię.
\item Używaj wcięć
\item Zwróć uwagę na szerokość tekstu
\item Używaj nawiasów do oznaczenia kolejności działań
\item Wykorzystuj puste linie do oddzielenia bloków kodu
\item Pisz komentarze
\end{itemize}

\end{frame}

\begin{frame}
\frametitle{Klasy i obiekty}

Obiekt łączy w sobie:

\begin{itemize}
  \item \emph{stan} - konkretny zestaw danych
  \item \emph{zachowanie} - kod działający na tych danych
  \item obiekt = dane + kod
\end{itemize}

\vspace{8mm}
Klasa to sztanca, za pomocą której formowane są obiekty.

\vspace{8mm}
Każda klasa definiuje nowy typ w programie.
\end{frame}

\begin{frame}[fragile]
\frametitle{Minimalny przykład}

\begin{lstlisting}
class Point
{
};

int main()
{
    Point p;
}
\end{lstlisting}

Oczywiście obiekty tak zadeklarowanej klasy Point są puste i nie można nic sensownego z nimi zrobić. Ale powyższy kod się skompiluje.

\end{frame}

\begin{frame}

\begin{itemize}
\item Klasa składa się ze składowych (\emph{members}).
\item Składowymi są albo dane (zmienne) albo metody (funkcje).
\item Dane przechowują stan klasy; metody pozwalają ten stan modyfikować (określają zachowanie klasy)
\item Można tworzyć wiele obiektów z danej klasy. Każdy ma swój zestaw zmiennych (danych - stan).
\item Wywoływanie metod obiektu zmienia stan wyłącznie tego obiektu, nie zmieniając stanów innych obiektów z danej klasy (są wyjątki - składowe statyczne).
\item Pojedyncze obiekty nazywamy instancjami danej klasy.
\item Klasa nie jest obiektem.
\end{itemize}
\end{frame}

\begin{frame}[fragile]
\frametitle{Dane i ich inicjalizacja}

\begin{lstlisting}
class Point
{
    double x,y;
    
public:
    Point(double _x, double _y) {
    	x = _x;
	y = _y;
    }
};
\end{lstlisting}

\begin{itemize}
\item Sekcje publiczne i prywatne
\item Zazwyczaj chcemy ukryć zmienne klasy
\end{itemize}

\end{frame}

\begin{frame}
\frametitle{Konstruktory i destruktory}

\begin{itemize}
\item Konstruktor tworzy nową instancję klasy.
\item Konstruktor może mieć argumenty, ale nie musi. Konstruktor bezargumentowy to konstruktor domyślny.
\item Konstruktor nie zwraca żadnej wartości.
\item Może być wiele konstruktorów, zawsze jest przynajmniej jeden.
\end{itemize}

\begin{itemize}
\item Destruktor "sprząta" po obiekcie.
\item Nie ma parametrów i nie zwraca wartości.
\item Jest tylko jeden destruktor.
\end{itemize}

Destruktory zaczną nam  być potrzebne dopiero, gdy zaczniemy umieszczać obiekty na stercie.
\end{frame}

\begin{frame}[fragile]
\frametitle{Jak modyfikować wartości zmiennych?}

Zmienne są w sekcji prywatnej - jak je modyfikować?

\begin{lstlisting}
class Point
{
	double x,y;
    
public:
	double getX() { return x; }
	double getY() { return y; }
	
	void setX(double _x) { x = _x; }
	void setY(double _y) { y = _y; }
};
\end{lstlisting}

\end{frame}

\begin{frame}
\frametitle{Metody}

\begin{itemize}
\item Metody dostępowe

\vspace{2mm}
Umożliwiają odczytanie stanu obiektu.
\item Metody ustawiające

\vspace{2mm}
Umożliwiają ustawienie stanu obiektu.

\item Metody modyfikujące

\vspace{2mm}
Metody umożliwiające działanie na obiekcie.

\item Modyfikatory private:, protected: i public:
\end{itemize}

\end{frame}

\begin{frame}

\begin{center}
\Large Projektowanie klas
\end{center}
\end{frame}

\begin{frame}
\frametitle{Myślenie obiektowe}

\begin{itemize}
\item Uchwycenie istotnych cech obiektu (abstraction)

\vspace{8mm}
\item Oddzielenie funkcjonalności od implementacji (encapsulation)
\end{itemize}

\end{frame}

\begin{frame}[fragile]
\frametitle{Proste na płaszczyźnie}

Chcemy utworzyć klasę, której obiekty reprezentują proste na płaszczyźnie.

\vspace{2mm}
Chcemy móc sprawdzać równoległość i wyznaczać punkt przecięcia dwóch prostych.

\vspace{2mm}
\begin{lstlisting}
class Line
{
public:
	bool parallelTo(Line);
	Point intersectionWith(Line);
};
\end{lstlisting}

(Powyższa klasa jest niepełna - brak implementacji metod i zmiennych klasy. Uwaga na różnicę pomiędzy deklaracją i definicją funkcji/metody.)
\end{frame}

\begin{frame}
\frametitle{Jak implementować klasę Line?}

Musimy zdecydować, jak reprezentować nasze proste.
\vspace{6mm}

\begin{enumerate}
\item Pamiętając współczynniki $A, B, C$ równania $Ax + By + C = 0$?

\vspace{4mm}
\item Pamiętając parę punktów $P, Q$, przez które przechodzi prosta?

\vspace{4mm}
\item Pamiętając punkt zaczepienia $P$ i wektor kierunkowy $v$ prostej?
\end{enumerate}

\end{frame}

\begin{frame}
\frametitle{Jak implementować klasę Line?}

Zwróć uwagę, że już samo pytanie o \emph{równość} dwóch prostych nie jest łatwe w żadnej z tych reprezentacji.

\vspace{4mm}
Szczegóły implementacji metod parallelTo i intersectionWith są zupełnie inne w każdym przypadku.

\vspace{4mm}
Nie jest to jednak istotne dla użytkownika klasy (jak długo umie on tworzyć i modyfikować obiekty tej klasy).

\vspace{4mm}
Do klasy Line wrócimy na następnych zajęciach
\end{frame}


\begin{frame}

\begin{center}
\Large Implementacja klas
\end{center}
\end{frame}

\begin{frame}[fragile]
\frametitle{Implementacja klasy Point}

\begin{lstlisting}
class Point
{
    double x, y;

public:
    Point();
    Point(double, double);

    ~Point();

    void setX(double);
    void setY(double);

    double getX();
    double getY();

    friend istream& operator>>(istream&, Point&);
};

ostream& operator<<(ostream &, Point);
istream& operator>>(istream &, Point&);
\end{lstlisting}
\end{frame}

\begin{frame}
\frametitle{Wskaźniki kontra referencje}

Metody obiektu zawsze przekazują zmienne przez wartość.

\vspace{8mm}
By oszczędzić na kopiowaniu używamy referencji.

\end{frame}

\begin{frame}[fragile]
\frametitle{Operatory << i >>}

\begin{lstlisting}
ostream& operator<<(ostream &o, Point p)
{
    o << "(" << p.getX() << "," << p.getY() << ")";

    return o;
}

istream& operator>>(istream &i, Point &p)
{
    i >> p.x;
    i >> p.y;

    return i;
}
\end{lstlisting}

\end{frame}

\begin{frame}[fragile]{Zadanie}

\begin{lstlisting}
class Plane {
public:
	Plane(Point A, Point B, Point C);
	moveTo(Point P);
	distanceFrom(Point P);
	parallelTo(Plane &P);
};
\end{lstlisting}

\end{frame}

\end{document}
